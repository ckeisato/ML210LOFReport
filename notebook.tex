
% Default to the notebook output style

    


% Inherit from the specified cell style.




    
\documentclass[11pt]{article}

    
    
    \usepackage[T1]{fontenc}
    % Nicer default font (+ math font) than Computer Modern for most use cases
    \usepackage{mathpazo}

    % Basic figure setup, for now with no caption control since it's done
    % automatically by Pandoc (which extracts ![](path) syntax from Markdown).
    \usepackage{graphicx}
    % We will generate all images so they have a width \maxwidth. This means
    % that they will get their normal width if they fit onto the page, but
    % are scaled down if they would overflow the margins.
    \makeatletter
    \def\maxwidth{\ifdim\Gin@nat@width>\linewidth\linewidth
    \else\Gin@nat@width\fi}
    \makeatother
    \let\Oldincludegraphics\includegraphics
    % Set max figure width to be 80% of text width, for now hardcoded.
    \renewcommand{\includegraphics}[1]{\Oldincludegraphics[width=.8\maxwidth]{#1}}
    % Ensure that by default, figures have no caption (until we provide a
    % proper Figure object with a Caption API and a way to capture that
    % in the conversion process - todo).
    \usepackage{caption}
    \DeclareCaptionLabelFormat{nolabel}{}
    \captionsetup{labelformat=nolabel}

    \usepackage{adjustbox} % Used to constrain images to a maximum size 
    \usepackage{xcolor} % Allow colors to be defined
    \usepackage{enumerate} % Needed for markdown enumerations to work
    \usepackage{geometry} % Used to adjust the document margins
    \usepackage{amsmath} % Equations
    \usepackage{amssymb} % Equations
    \usepackage{textcomp} % defines textquotesingle
    % Hack from http://tex.stackexchange.com/a/47451/13684:
    \AtBeginDocument{%
        \def\PYZsq{\textquotesingle}% Upright quotes in Pygmentized code
    }
    \usepackage{upquote} % Upright quotes for verbatim code
    \usepackage{eurosym} % defines \euro
    \usepackage[mathletters]{ucs} % Extended unicode (utf-8) support
    \usepackage[utf8x]{inputenc} % Allow utf-8 characters in the tex document
    \usepackage{fancyvrb} % verbatim replacement that allows latex
    \usepackage{grffile} % extends the file name processing of package graphics 
                         % to support a larger range 
    % The hyperref package gives us a pdf with properly built
    % internal navigation ('pdf bookmarks' for the table of contents,
    % internal cross-reference links, web links for URLs, etc.)
    \usepackage{hyperref}
    \usepackage{longtable} % longtable support required by pandoc >1.10
    \usepackage{booktabs}  % table support for pandoc > 1.12.2
    \usepackage[inline]{enumitem} % IRkernel/repr support (it uses the enumerate* environment)
    \usepackage[normalem]{ulem} % ulem is needed to support strikethroughs (\sout)
                                % normalem makes italics be italics, not underlines
    

    
    
    % Colors for the hyperref package
    \definecolor{urlcolor}{rgb}{0,.145,.698}
    \definecolor{linkcolor}{rgb}{.71,0.21,0.01}
    \definecolor{citecolor}{rgb}{.12,.54,.11}

    % ANSI colors
    \definecolor{ansi-black}{HTML}{3E424D}
    \definecolor{ansi-black-intense}{HTML}{282C36}
    \definecolor{ansi-red}{HTML}{E75C58}
    \definecolor{ansi-red-intense}{HTML}{B22B31}
    \definecolor{ansi-green}{HTML}{00A250}
    \definecolor{ansi-green-intense}{HTML}{007427}
    \definecolor{ansi-yellow}{HTML}{DDB62B}
    \definecolor{ansi-yellow-intense}{HTML}{B27D12}
    \definecolor{ansi-blue}{HTML}{208FFB}
    \definecolor{ansi-blue-intense}{HTML}{0065CA}
    \definecolor{ansi-magenta}{HTML}{D160C4}
    \definecolor{ansi-magenta-intense}{HTML}{A03196}
    \definecolor{ansi-cyan}{HTML}{60C6C8}
    \definecolor{ansi-cyan-intense}{HTML}{258F8F}
    \definecolor{ansi-white}{HTML}{C5C1B4}
    \definecolor{ansi-white-intense}{HTML}{A1A6B2}

    % commands and environments needed by pandoc snippets
    % extracted from the output of `pandoc -s`
    \providecommand{\tightlist}{%
      \setlength{\itemsep}{0pt}\setlength{\parskip}{0pt}}
    \DefineVerbatimEnvironment{Highlighting}{Verbatim}{commandchars=\\\{\}}
    % Add ',fontsize=\small' for more characters per line
    \newenvironment{Shaded}{}{}
    \newcommand{\KeywordTok}[1]{\textcolor[rgb]{0.00,0.44,0.13}{\textbf{{#1}}}}
    \newcommand{\DataTypeTok}[1]{\textcolor[rgb]{0.56,0.13,0.00}{{#1}}}
    \newcommand{\DecValTok}[1]{\textcolor[rgb]{0.25,0.63,0.44}{{#1}}}
    \newcommand{\BaseNTok}[1]{\textcolor[rgb]{0.25,0.63,0.44}{{#1}}}
    \newcommand{\FloatTok}[1]{\textcolor[rgb]{0.25,0.63,0.44}{{#1}}}
    \newcommand{\CharTok}[1]{\textcolor[rgb]{0.25,0.44,0.63}{{#1}}}
    \newcommand{\StringTok}[1]{\textcolor[rgb]{0.25,0.44,0.63}{{#1}}}
    \newcommand{\CommentTok}[1]{\textcolor[rgb]{0.38,0.63,0.69}{\textit{{#1}}}}
    \newcommand{\OtherTok}[1]{\textcolor[rgb]{0.00,0.44,0.13}{{#1}}}
    \newcommand{\AlertTok}[1]{\textcolor[rgb]{1.00,0.00,0.00}{\textbf{{#1}}}}
    \newcommand{\FunctionTok}[1]{\textcolor[rgb]{0.02,0.16,0.49}{{#1}}}
    \newcommand{\RegionMarkerTok}[1]{{#1}}
    \newcommand{\ErrorTok}[1]{\textcolor[rgb]{1.00,0.00,0.00}{\textbf{{#1}}}}
    \newcommand{\NormalTok}[1]{{#1}}
    
    % Additional commands for more recent versions of Pandoc
    \newcommand{\ConstantTok}[1]{\textcolor[rgb]{0.53,0.00,0.00}{{#1}}}
    \newcommand{\SpecialCharTok}[1]{\textcolor[rgb]{0.25,0.44,0.63}{{#1}}}
    \newcommand{\VerbatimStringTok}[1]{\textcolor[rgb]{0.25,0.44,0.63}{{#1}}}
    \newcommand{\SpecialStringTok}[1]{\textcolor[rgb]{0.73,0.40,0.53}{{#1}}}
    \newcommand{\ImportTok}[1]{{#1}}
    \newcommand{\DocumentationTok}[1]{\textcolor[rgb]{0.73,0.13,0.13}{\textit{{#1}}}}
    \newcommand{\AnnotationTok}[1]{\textcolor[rgb]{0.38,0.63,0.69}{\textbf{\textit{{#1}}}}}
    \newcommand{\CommentVarTok}[1]{\textcolor[rgb]{0.38,0.63,0.69}{\textbf{\textit{{#1}}}}}
    \newcommand{\VariableTok}[1]{\textcolor[rgb]{0.10,0.09,0.49}{{#1}}}
    \newcommand{\ControlFlowTok}[1]{\textcolor[rgb]{0.00,0.44,0.13}{\textbf{{#1}}}}
    \newcommand{\OperatorTok}[1]{\textcolor[rgb]{0.40,0.40,0.40}{{#1}}}
    \newcommand{\BuiltInTok}[1]{{#1}}
    \newcommand{\ExtensionTok}[1]{{#1}}
    \newcommand{\PreprocessorTok}[1]{\textcolor[rgb]{0.74,0.48,0.00}{{#1}}}
    \newcommand{\AttributeTok}[1]{\textcolor[rgb]{0.49,0.56,0.16}{{#1}}}
    \newcommand{\InformationTok}[1]{\textcolor[rgb]{0.38,0.63,0.69}{\textbf{\textit{{#1}}}}}
    \newcommand{\WarningTok}[1]{\textcolor[rgb]{0.38,0.63,0.69}{\textbf{\textit{{#1}}}}}
    
    
    % Define a nice break command that doesn't care if a line doesn't already
    % exist.
    \def\br{\hspace*{\fill} \\* }
    % Math Jax compatability definitions
    \def\gt{>}
    \def\lt{<}
    % Document parameters
    \title{index}
    
    
    

    % Pygments definitions
    
\makeatletter
\def\PY@reset{\let\PY@it=\relax \let\PY@bf=\relax%
    \let\PY@ul=\relax \let\PY@tc=\relax%
    \let\PY@bc=\relax \let\PY@ff=\relax}
\def\PY@tok#1{\csname PY@tok@#1\endcsname}
\def\PY@toks#1+{\ifx\relax#1\empty\else%
    \PY@tok{#1}\expandafter\PY@toks\fi}
\def\PY@do#1{\PY@bc{\PY@tc{\PY@ul{%
    \PY@it{\PY@bf{\PY@ff{#1}}}}}}}
\def\PY#1#2{\PY@reset\PY@toks#1+\relax+\PY@do{#2}}

\expandafter\def\csname PY@tok@w\endcsname{\def\PY@tc##1{\textcolor[rgb]{0.73,0.73,0.73}{##1}}}
\expandafter\def\csname PY@tok@c\endcsname{\let\PY@it=\textit\def\PY@tc##1{\textcolor[rgb]{0.25,0.50,0.50}{##1}}}
\expandafter\def\csname PY@tok@cp\endcsname{\def\PY@tc##1{\textcolor[rgb]{0.74,0.48,0.00}{##1}}}
\expandafter\def\csname PY@tok@k\endcsname{\let\PY@bf=\textbf\def\PY@tc##1{\textcolor[rgb]{0.00,0.50,0.00}{##1}}}
\expandafter\def\csname PY@tok@kp\endcsname{\def\PY@tc##1{\textcolor[rgb]{0.00,0.50,0.00}{##1}}}
\expandafter\def\csname PY@tok@kt\endcsname{\def\PY@tc##1{\textcolor[rgb]{0.69,0.00,0.25}{##1}}}
\expandafter\def\csname PY@tok@o\endcsname{\def\PY@tc##1{\textcolor[rgb]{0.40,0.40,0.40}{##1}}}
\expandafter\def\csname PY@tok@ow\endcsname{\let\PY@bf=\textbf\def\PY@tc##1{\textcolor[rgb]{0.67,0.13,1.00}{##1}}}
\expandafter\def\csname PY@tok@nb\endcsname{\def\PY@tc##1{\textcolor[rgb]{0.00,0.50,0.00}{##1}}}
\expandafter\def\csname PY@tok@nf\endcsname{\def\PY@tc##1{\textcolor[rgb]{0.00,0.00,1.00}{##1}}}
\expandafter\def\csname PY@tok@nc\endcsname{\let\PY@bf=\textbf\def\PY@tc##1{\textcolor[rgb]{0.00,0.00,1.00}{##1}}}
\expandafter\def\csname PY@tok@nn\endcsname{\let\PY@bf=\textbf\def\PY@tc##1{\textcolor[rgb]{0.00,0.00,1.00}{##1}}}
\expandafter\def\csname PY@tok@ne\endcsname{\let\PY@bf=\textbf\def\PY@tc##1{\textcolor[rgb]{0.82,0.25,0.23}{##1}}}
\expandafter\def\csname PY@tok@nv\endcsname{\def\PY@tc##1{\textcolor[rgb]{0.10,0.09,0.49}{##1}}}
\expandafter\def\csname PY@tok@no\endcsname{\def\PY@tc##1{\textcolor[rgb]{0.53,0.00,0.00}{##1}}}
\expandafter\def\csname PY@tok@nl\endcsname{\def\PY@tc##1{\textcolor[rgb]{0.63,0.63,0.00}{##1}}}
\expandafter\def\csname PY@tok@ni\endcsname{\let\PY@bf=\textbf\def\PY@tc##1{\textcolor[rgb]{0.60,0.60,0.60}{##1}}}
\expandafter\def\csname PY@tok@na\endcsname{\def\PY@tc##1{\textcolor[rgb]{0.49,0.56,0.16}{##1}}}
\expandafter\def\csname PY@tok@nt\endcsname{\let\PY@bf=\textbf\def\PY@tc##1{\textcolor[rgb]{0.00,0.50,0.00}{##1}}}
\expandafter\def\csname PY@tok@nd\endcsname{\def\PY@tc##1{\textcolor[rgb]{0.67,0.13,1.00}{##1}}}
\expandafter\def\csname PY@tok@s\endcsname{\def\PY@tc##1{\textcolor[rgb]{0.73,0.13,0.13}{##1}}}
\expandafter\def\csname PY@tok@sd\endcsname{\let\PY@it=\textit\def\PY@tc##1{\textcolor[rgb]{0.73,0.13,0.13}{##1}}}
\expandafter\def\csname PY@tok@si\endcsname{\let\PY@bf=\textbf\def\PY@tc##1{\textcolor[rgb]{0.73,0.40,0.53}{##1}}}
\expandafter\def\csname PY@tok@se\endcsname{\let\PY@bf=\textbf\def\PY@tc##1{\textcolor[rgb]{0.73,0.40,0.13}{##1}}}
\expandafter\def\csname PY@tok@sr\endcsname{\def\PY@tc##1{\textcolor[rgb]{0.73,0.40,0.53}{##1}}}
\expandafter\def\csname PY@tok@ss\endcsname{\def\PY@tc##1{\textcolor[rgb]{0.10,0.09,0.49}{##1}}}
\expandafter\def\csname PY@tok@sx\endcsname{\def\PY@tc##1{\textcolor[rgb]{0.00,0.50,0.00}{##1}}}
\expandafter\def\csname PY@tok@m\endcsname{\def\PY@tc##1{\textcolor[rgb]{0.40,0.40,0.40}{##1}}}
\expandafter\def\csname PY@tok@gh\endcsname{\let\PY@bf=\textbf\def\PY@tc##1{\textcolor[rgb]{0.00,0.00,0.50}{##1}}}
\expandafter\def\csname PY@tok@gu\endcsname{\let\PY@bf=\textbf\def\PY@tc##1{\textcolor[rgb]{0.50,0.00,0.50}{##1}}}
\expandafter\def\csname PY@tok@gd\endcsname{\def\PY@tc##1{\textcolor[rgb]{0.63,0.00,0.00}{##1}}}
\expandafter\def\csname PY@tok@gi\endcsname{\def\PY@tc##1{\textcolor[rgb]{0.00,0.63,0.00}{##1}}}
\expandafter\def\csname PY@tok@gr\endcsname{\def\PY@tc##1{\textcolor[rgb]{1.00,0.00,0.00}{##1}}}
\expandafter\def\csname PY@tok@ge\endcsname{\let\PY@it=\textit}
\expandafter\def\csname PY@tok@gs\endcsname{\let\PY@bf=\textbf}
\expandafter\def\csname PY@tok@gp\endcsname{\let\PY@bf=\textbf\def\PY@tc##1{\textcolor[rgb]{0.00,0.00,0.50}{##1}}}
\expandafter\def\csname PY@tok@go\endcsname{\def\PY@tc##1{\textcolor[rgb]{0.53,0.53,0.53}{##1}}}
\expandafter\def\csname PY@tok@gt\endcsname{\def\PY@tc##1{\textcolor[rgb]{0.00,0.27,0.87}{##1}}}
\expandafter\def\csname PY@tok@err\endcsname{\def\PY@bc##1{\setlength{\fboxsep}{0pt}\fcolorbox[rgb]{1.00,0.00,0.00}{1,1,1}{\strut ##1}}}
\expandafter\def\csname PY@tok@kc\endcsname{\let\PY@bf=\textbf\def\PY@tc##1{\textcolor[rgb]{0.00,0.50,0.00}{##1}}}
\expandafter\def\csname PY@tok@kd\endcsname{\let\PY@bf=\textbf\def\PY@tc##1{\textcolor[rgb]{0.00,0.50,0.00}{##1}}}
\expandafter\def\csname PY@tok@kn\endcsname{\let\PY@bf=\textbf\def\PY@tc##1{\textcolor[rgb]{0.00,0.50,0.00}{##1}}}
\expandafter\def\csname PY@tok@kr\endcsname{\let\PY@bf=\textbf\def\PY@tc##1{\textcolor[rgb]{0.00,0.50,0.00}{##1}}}
\expandafter\def\csname PY@tok@bp\endcsname{\def\PY@tc##1{\textcolor[rgb]{0.00,0.50,0.00}{##1}}}
\expandafter\def\csname PY@tok@fm\endcsname{\def\PY@tc##1{\textcolor[rgb]{0.00,0.00,1.00}{##1}}}
\expandafter\def\csname PY@tok@vc\endcsname{\def\PY@tc##1{\textcolor[rgb]{0.10,0.09,0.49}{##1}}}
\expandafter\def\csname PY@tok@vg\endcsname{\def\PY@tc##1{\textcolor[rgb]{0.10,0.09,0.49}{##1}}}
\expandafter\def\csname PY@tok@vi\endcsname{\def\PY@tc##1{\textcolor[rgb]{0.10,0.09,0.49}{##1}}}
\expandafter\def\csname PY@tok@vm\endcsname{\def\PY@tc##1{\textcolor[rgb]{0.10,0.09,0.49}{##1}}}
\expandafter\def\csname PY@tok@sa\endcsname{\def\PY@tc##1{\textcolor[rgb]{0.73,0.13,0.13}{##1}}}
\expandafter\def\csname PY@tok@sb\endcsname{\def\PY@tc##1{\textcolor[rgb]{0.73,0.13,0.13}{##1}}}
\expandafter\def\csname PY@tok@sc\endcsname{\def\PY@tc##1{\textcolor[rgb]{0.73,0.13,0.13}{##1}}}
\expandafter\def\csname PY@tok@dl\endcsname{\def\PY@tc##1{\textcolor[rgb]{0.73,0.13,0.13}{##1}}}
\expandafter\def\csname PY@tok@s2\endcsname{\def\PY@tc##1{\textcolor[rgb]{0.73,0.13,0.13}{##1}}}
\expandafter\def\csname PY@tok@sh\endcsname{\def\PY@tc##1{\textcolor[rgb]{0.73,0.13,0.13}{##1}}}
\expandafter\def\csname PY@tok@s1\endcsname{\def\PY@tc##1{\textcolor[rgb]{0.73,0.13,0.13}{##1}}}
\expandafter\def\csname PY@tok@mb\endcsname{\def\PY@tc##1{\textcolor[rgb]{0.40,0.40,0.40}{##1}}}
\expandafter\def\csname PY@tok@mf\endcsname{\def\PY@tc##1{\textcolor[rgb]{0.40,0.40,0.40}{##1}}}
\expandafter\def\csname PY@tok@mh\endcsname{\def\PY@tc##1{\textcolor[rgb]{0.40,0.40,0.40}{##1}}}
\expandafter\def\csname PY@tok@mi\endcsname{\def\PY@tc##1{\textcolor[rgb]{0.40,0.40,0.40}{##1}}}
\expandafter\def\csname PY@tok@il\endcsname{\def\PY@tc##1{\textcolor[rgb]{0.40,0.40,0.40}{##1}}}
\expandafter\def\csname PY@tok@mo\endcsname{\def\PY@tc##1{\textcolor[rgb]{0.40,0.40,0.40}{##1}}}
\expandafter\def\csname PY@tok@ch\endcsname{\let\PY@it=\textit\def\PY@tc##1{\textcolor[rgb]{0.25,0.50,0.50}{##1}}}
\expandafter\def\csname PY@tok@cm\endcsname{\let\PY@it=\textit\def\PY@tc##1{\textcolor[rgb]{0.25,0.50,0.50}{##1}}}
\expandafter\def\csname PY@tok@cpf\endcsname{\let\PY@it=\textit\def\PY@tc##1{\textcolor[rgb]{0.25,0.50,0.50}{##1}}}
\expandafter\def\csname PY@tok@c1\endcsname{\let\PY@it=\textit\def\PY@tc##1{\textcolor[rgb]{0.25,0.50,0.50}{##1}}}
\expandafter\def\csname PY@tok@cs\endcsname{\let\PY@it=\textit\def\PY@tc##1{\textcolor[rgb]{0.25,0.50,0.50}{##1}}}

\def\PYZbs{\char`\\}
\def\PYZus{\char`\_}
\def\PYZob{\char`\{}
\def\PYZcb{\char`\}}
\def\PYZca{\char`\^}
\def\PYZam{\char`\&}
\def\PYZlt{\char`\<}
\def\PYZgt{\char`\>}
\def\PYZsh{\char`\#}
\def\PYZpc{\char`\%}
\def\PYZdl{\char`\$}
\def\PYZhy{\char`\-}
\def\PYZsq{\char`\'}
\def\PYZdq{\char`\"}
\def\PYZti{\char`\~}
% for compatibility with earlier versions
\def\PYZat{@}
\def\PYZlb{[}
\def\PYZrb{]}
\makeatother


    % Exact colors from NB
    \definecolor{incolor}{rgb}{0.0, 0.0, 0.5}
    \definecolor{outcolor}{rgb}{0.545, 0.0, 0.0}



    
    % Prevent overflowing lines due to hard-to-break entities
    \sloppy 
    % Setup hyperref package
    \hypersetup{
      breaklinks=true,  % so long urls are correctly broken across lines
      colorlinks=true,
      urlcolor=urlcolor,
      linkcolor=linkcolor,
      citecolor=citecolor,
      }
    % Slightly bigger margins than the latex defaults
    
    \geometry{verbose,tmargin=1in,bmargin=1in,lmargin=1in,rmargin=1in}
    
    

    \begin{document}
    
    
    \maketitle
    
    

    
    \subsection{MLearn 210: Final Project - Local Outlier Factor
Report}\label{mlearn-210-final-project---local-outlier-factor-report}

\paragraph{December 10, 2018}\label{december-10-2018}

\paragraph{Kei Sato}\label{kei-sato}

    Local Outlier Factor (LOF) is unsupervised learning algorithm for
anomaly detection that utilizes the local density of a data point. The
algorithm calculates the local density of each point, a measure of how
crowded the area around the observation is, and gives that datapoint an
LOF score. The LOF score is the ratio of the data point's density and
the average density of x neighbors (x is a parameter). If a data point
has an LOF greater than 1, then its neighbors have a higher local
density than the data point and it is an outlier. While there are other
anomaly detection algorithms, LOF addresses the problem of a data point
being a local outlier, which is an a outlier relative to its surrounding
points instead of relative to the entire dataset. This maybe be useful
to flag data points that would otherwise not be considered outliers
because their values may be close to the global average feature
attribute values, but the points themselves are not in any clusters.

To determine the LOF of a datapoint, the local reachability density has
to be calculated using the using k-distance, reachability distance, and
min-points.

    \paragraph{The k-distance of a data point is defined
as:}\label{the-k-distance-of-a-data-point-is-defined-as}

\begin{verbatim}
For any positive integer k, the k-distance of object p, denoted as k-distance(p), is defined as the distance d(p,o) between p and an object o ∈ D such that:                
    (i) for at least k objects o’∈D \ {p} it holds that d(p,o’) ≤ d(p,o), and
    (ii) for at most k-1 objects o’∈D\{p} it holds that d(p,o’) < d(p,o).
\end{verbatim}

1

    Given this definition, if k = 3, the k-distance of point p is the
distance from p to its third closest neighbor

    \paragraph{The reachability distance is defined
as:}\label{the-reachability-distance-is-defined-as}

\begin{verbatim}
(reachability distance of an object p with respect to object o)
        Let k be a natural number. The reachability distance of object p with respect to object o is defined as
        reach-distk(p, o) = max { k-distance(o), d(p, o) }.
\end{verbatim}

2

    The reachability distance between datapoint p with respect to datapoint
o is either the actual distance between two data points or the k
distance of p, whichever is larger. If p is within the k-distance
neighborhood of o, then its reachability distance with respect to o is
the k-distance of o. If k = 3, then for the 2nd nearest neighbor of
point p, the reachability distance of that datapoint with respect to p
will be the k-distance of p (the distance from p to its 3rd nearest
neighbor). The reachability distance is a smoothing factor meant to
decrease sensitivity in the algorithm.

We later define minpoints, which is a parameter to specify the volume
for the local reachability density. Minpoints is greater than or equal
to k. Simplified LOF is a variation on LOF for which k is the same as
minpoints, and we always use the actual distance between points as the
reachability distance. Simplified LOF is slightly more sensitive and may
give less consistent results with regard to inliers.

    \paragraph{The local reachability
density:}\label{the-local-reachability-density}

The local reachability density of object p

\[lrd_{MinPts}(p) = 1/{\frac{\sum\limits_{o\in{N_{MinPts(p)}}}{reach-dist_{MinPts}(o,p)}}{| N_{MinPts}(p) |}}\]
3

Given a value for minpoints, the local reachability density for point p
is the inverse of it average reach distance over minpoints number of
neighbors.

    \paragraph{Local Outlier Factor}\label{local-outlier-factor}

The local outlier factor for a point p

\[LOF_{MinPts}(p) = \frac{\sum\limits_{o\in{N_{MinPts}(p)}}{\frac{lrd_{MinPts}(o)}{lrd_{MinPts}(p)}}}{|N_{MinPts}(p)|}\]
4

    With this definition, the LOF of a point p is the average ratio of it's
local reachability density and that of its minpoints neighbors. If a
data point has an LOF around 1, then its density is similar to its
neighbors, while if its LOF greater than 1, then its density is much
lower than that of its neighbors and it may be an outlier.

The LOF has been applied to the sklean Boston dataset below to
demonstrate the use of LOF. The Boston dataset is a collection of
attributes of different towns around Boston, such as the average age or
size of houses there, and the median home price of the town. For the
following demo, the predictor value is the per capita crime rate by town
and the target value is the median value of owner-occupied homes in the
\$1000s.

    \begin{Verbatim}[commandchars=\\\{\}]
{\color{incolor}In [{\color{incolor}4}]:} \PY{k+kn}{from} \PY{n+nn}{sklearn}\PY{n+nn}{.}\PY{n+nn}{datasets} \PY{k}{import} \PY{n}{load\PYZus{}boston}
        \PY{k+kn}{import} \PY{n+nn}{pandas} \PY{k}{as} \PY{n+nn}{pd}
        \PY{k+kn}{import} \PY{n+nn}{numpy} \PY{k}{as} \PY{n+nn}{np}
        \PY{k+kn}{import} \PY{n+nn}{matplotlib}\PY{n+nn}{.}\PY{n+nn}{pyplot} \PY{k}{as} \PY{n+nn}{plt}
        
        \PY{n}{boston} \PY{o}{=} \PY{n}{load\PYZus{}boston}\PY{p}{(}\PY{p}{)}
        \PY{n}{data} \PY{o}{=} \PY{n}{pd}\PY{o}{.}\PY{n}{DataFrame}\PY{p}{(}\PY{n}{boston}\PY{o}{.}\PY{n}{data}\PY{p}{,} \PY{n}{columns} \PY{o}{=} \PY{n}{boston}\PY{p}{[}\PY{l+s+s1}{\PYZsq{}}\PY{l+s+s1}{feature\PYZus{}names}\PY{l+s+s1}{\PYZsq{}}\PY{p}{]}\PY{p}{)}
        \PY{n}{data}\PY{p}{[}\PY{l+s+s1}{\PYZsq{}}\PY{l+s+s1}{MEDV}\PY{l+s+s1}{\PYZsq{}}\PY{p}{]} \PY{o}{=} \PY{n}{boston}\PY{o}{.}\PY{n}{target}
        \PY{n}{data}\PY{o}{.}\PY{n}{head}\PY{p}{(}\PY{p}{)}
        
        \PY{n}{plt}\PY{o}{.}\PY{n}{figure}\PY{p}{(}\PY{n}{figsize}\PY{o}{=}\PY{p}{(}\PY{l+m+mi}{10}\PY{p}{,} \PY{l+m+mi}{10}\PY{p}{)}\PY{p}{)}
        \PY{n}{plt}\PY{o}{.}\PY{n}{scatter}\PY{p}{(}\PY{n}{data}\PY{o}{.}\PY{n}{loc}\PY{p}{[}\PY{p}{:}\PY{p}{,} \PY{l+s+s1}{\PYZsq{}}\PY{l+s+s1}{CRIM}\PY{l+s+s1}{\PYZsq{}}\PY{p}{]}\PY{p}{,} \PY{n}{data}\PY{o}{.}\PY{n}{loc}\PY{p}{[}\PY{p}{:}\PY{p}{,} \PY{l+s+s1}{\PYZsq{}}\PY{l+s+s1}{MEDV}\PY{l+s+s1}{\PYZsq{}}\PY{p}{]}\PY{p}{,} \PY{n}{marker}\PY{o}{=}\PY{l+s+s1}{\PYZsq{}}\PY{l+s+s1}{o}\PY{l+s+s1}{\PYZsq{}}\PY{p}{,} \PY{n}{s}\PY{o}{=}\PY{l+m+mi}{25}\PY{p}{)}
        \PY{n}{plt}\PY{o}{.}\PY{n}{xlabel}\PY{p}{(}\PY{l+s+s2}{\PYZdq{}}\PY{l+s+s2}{per capita crime rate by town}\PY{l+s+s2}{\PYZdq{}}\PY{p}{)}
        \PY{n}{plt}\PY{o}{.}\PY{n}{ylabel}\PY{p}{(}\PY{l+s+s2}{\PYZdq{}}\PY{l+s+s2}{Median value of owner\PYZhy{}occupied homes in \PYZdl{}1000’s}\PY{l+s+s2}{\PYZdq{}}\PY{p}{)}
        \PY{n}{plt}\PY{o}{.}\PY{n}{show}\PY{p}{(}\PY{p}{)}
\end{Verbatim}


    \begin{center}
    \adjustimage{max size={0.9\linewidth}{0.9\paperheight}}{output_9_0.png}
    \end{center}
    { \hspace*{\fill} \\}
    
    From the plot we can see that there is an inverse relationship between
the median home price and the per capita crime rate in a town. Towards
the lower right of the graph are some towns with exceptionally high
crime rates, which we would consider outliers. Red dots are the outliers
while blue dots are the inliers

    \begin{Verbatim}[commandchars=\\\{\}]
{\color{incolor}In [{\color{incolor}2}]:} \PY{k+kn}{from} \PY{n+nn}{sklearn}\PY{n+nn}{.}\PY{n+nn}{neighbors} \PY{k}{import} \PY{n}{LocalOutlierFactor}
        
        \PY{n}{clf} \PY{o}{=} \PY{n}{LocalOutlierFactor}\PY{p}{(}\PY{n}{n\PYZus{}neighbors}\PY{o}{=}\PY{l+m+mi}{5}\PY{p}{)}
        \PY{n}{prediction} \PY{o}{=} \PY{n}{clf}\PY{o}{.}\PY{n}{fit\PYZus{}predict}\PY{p}{(}\PY{n}{data}\PY{p}{[}\PY{p}{[}\PY{l+s+s1}{\PYZsq{}}\PY{l+s+s1}{CRIM}\PY{l+s+s1}{\PYZsq{}}\PY{p}{,} \PY{l+s+s1}{\PYZsq{}}\PY{l+s+s1}{MEDV}\PY{l+s+s1}{\PYZsq{}}\PY{p}{]}\PY{p}{]}\PY{p}{)}
        
        
        \PY{n}{inliers\PYZus{}array} \PY{o}{=} \PY{n+nb}{list}\PY{p}{(}\PY{n+nb}{map}\PY{p}{(}\PY{k}{lambda} \PY{n}{x}\PY{p}{:} \PY{k+kc}{True} \PY{k}{if} \PY{n}{x} \PY{o}{==} \PY{l+m+mi}{1} \PY{k}{else} \PY{k+kc}{False}\PY{p}{,} \PY{n}{prediction}\PY{p}{)}\PY{p}{)}
        \PY{n}{outliers\PYZus{}array} \PY{o}{=} \PY{n+nb}{list}\PY{p}{(}\PY{n+nb}{map}\PY{p}{(}\PY{k}{lambda} \PY{n}{x}\PY{p}{:} \PY{k+kc}{False} \PY{k}{if} \PY{n}{x} \PY{o}{==} \PY{l+m+mi}{1} \PY{k}{else} \PY{k+kc}{True}\PY{p}{,} \PY{n}{prediction}\PY{p}{)}\PY{p}{)}
        
        \PY{n}{plt}\PY{o}{.}\PY{n}{figure}\PY{p}{(}\PY{n}{figsize}\PY{o}{=}\PY{p}{(}\PY{l+m+mi}{10}\PY{p}{,} \PY{l+m+mi}{10}\PY{p}{)}\PY{p}{)}
        \PY{n}{plt}\PY{o}{.}\PY{n}{scatter}\PY{p}{(}\PY{n}{data}\PY{p}{[}\PY{n}{inliers\PYZus{}array}\PY{p}{]}\PY{o}{.}\PY{n}{loc}\PY{p}{[}\PY{p}{:}\PY{p}{,} \PY{l+s+s1}{\PYZsq{}}\PY{l+s+s1}{CRIM}\PY{l+s+s1}{\PYZsq{}}\PY{p}{]}\PY{p}{,} \PY{n}{data}\PY{p}{[}\PY{n}{inliers\PYZus{}array}\PY{p}{]}\PY{o}{.}\PY{n}{loc}\PY{p}{[}\PY{p}{:}\PY{p}{,} \PY{l+s+s1}{\PYZsq{}}\PY{l+s+s1}{MEDV}\PY{l+s+s1}{\PYZsq{}}\PY{p}{]}\PY{p}{,}
                    \PY{n}{c}\PY{o}{=}\PY{l+s+s1}{\PYZsq{}}\PY{l+s+s1}{b}\PY{l+s+s1}{\PYZsq{}}\PY{p}{,} \PY{n}{marker}\PY{o}{=}\PY{l+s+s1}{\PYZsq{}}\PY{l+s+s1}{o}\PY{l+s+s1}{\PYZsq{}}\PY{p}{,} \PY{n}{s}\PY{o}{=}\PY{l+m+mi}{25}\PY{p}{,} \PY{n}{label}\PY{o}{=}\PY{l+s+s1}{\PYZsq{}}\PY{l+s+s1}{inliers}\PY{l+s+s1}{\PYZsq{}}\PY{p}{)}
        \PY{n}{plt}\PY{o}{.}\PY{n}{scatter}\PY{p}{(}\PY{n}{data}\PY{p}{[}\PY{n}{outliers\PYZus{}array}\PY{p}{]}\PY{o}{.}\PY{n}{loc}\PY{p}{[}\PY{p}{:}\PY{p}{,} \PY{l+s+s1}{\PYZsq{}}\PY{l+s+s1}{CRIM}\PY{l+s+s1}{\PYZsq{}}\PY{p}{]}\PY{p}{,} \PY{n}{data}\PY{p}{[}\PY{n}{outliers\PYZus{}array}\PY{p}{]}\PY{o}{.}\PY{n}{loc}\PY{p}{[}\PY{p}{:}\PY{p}{,} \PY{l+s+s1}{\PYZsq{}}\PY{l+s+s1}{MEDV}\PY{l+s+s1}{\PYZsq{}}\PY{p}{]}\PY{p}{,}
                    \PY{n}{c}\PY{o}{=}\PY{l+s+s1}{\PYZsq{}}\PY{l+s+s1}{r}\PY{l+s+s1}{\PYZsq{}}\PY{p}{,} \PY{n}{marker}\PY{o}{=}\PY{l+s+s1}{\PYZsq{}}\PY{l+s+s1}{o}\PY{l+s+s1}{\PYZsq{}}\PY{p}{,} \PY{n}{s}\PY{o}{=}\PY{l+m+mi}{25}\PY{p}{,} \PY{n}{label}\PY{o}{=}\PY{l+s+s1}{\PYZsq{}}\PY{l+s+s1}{outliers}\PY{l+s+s1}{\PYZsq{}}\PY{p}{)}
        
        \PY{n}{plt}\PY{o}{.}\PY{n}{xlabel}\PY{p}{(}\PY{l+s+s2}{\PYZdq{}}\PY{l+s+s2}{per capita crime rate by town}\PY{l+s+s2}{\PYZdq{}}\PY{p}{)}
        \PY{n}{plt}\PY{o}{.}\PY{n}{ylabel}\PY{p}{(}\PY{l+s+s2}{\PYZdq{}}\PY{l+s+s2}{Median value of owner\PYZhy{}occupied homes in \PYZdl{}1000’s}\PY{l+s+s2}{\PYZdq{}}\PY{p}{)}
        
        \PY{n}{legend} \PY{o}{=} \PY{n}{plt}\PY{o}{.}\PY{n}{legend}\PY{p}{(}\PY{n}{loc}\PY{o}{=}\PY{l+s+s1}{\PYZsq{}}\PY{l+s+s1}{upper right}\PY{l+s+s1}{\PYZsq{}}\PY{p}{)}
        \PY{n}{legend}\PY{o}{.}\PY{n}{legendHandles}\PY{p}{[}\PY{l+m+mi}{0}\PY{p}{]}\PY{o}{.}\PY{n}{\PYZus{}sizes} \PY{o}{=} \PY{p}{[}\PY{l+m+mi}{100}\PY{p}{]}
        \PY{n}{legend}\PY{o}{.}\PY{n}{legendHandles}\PY{p}{[}\PY{l+m+mi}{1}\PY{p}{]}\PY{o}{.}\PY{n}{\PYZus{}sizes} \PY{o}{=} \PY{p}{[}\PY{l+m+mi}{100}\PY{p}{]}
        
        \PY{n}{X\PYZus{}scores} \PY{o}{=} \PY{n}{clf}\PY{o}{.}\PY{n}{negative\PYZus{}outlier\PYZus{}factor\PYZus{}}
        \PY{n}{radius} \PY{o}{=} \PY{p}{(}\PY{n}{X\PYZus{}scores}\PY{o}{.}\PY{n}{max}\PY{p}{(}\PY{p}{)} \PY{o}{\PYZhy{}} \PY{n}{X\PYZus{}scores}\PY{p}{)} \PY{o}{/} \PY{p}{(}\PY{n}{X\PYZus{}scores}\PY{o}{.}\PY{n}{max}\PY{p}{(}\PY{p}{)} \PY{o}{\PYZhy{}} \PY{n}{X\PYZus{}scores}\PY{o}{.}\PY{n}{min}\PY{p}{(}\PY{p}{)}\PY{p}{)}
        \PY{n}{plt}\PY{o}{.}\PY{n}{show}\PY{p}{(}\PY{p}{)}
\end{Verbatim}


    \begin{center}
    \adjustimage{max size={0.9\linewidth}{0.9\paperheight}}{output_11_0.png}
    \end{center}
    { \hspace*{\fill} \\}
    
    This example shows the Sklearn LOF predictor run against the dataset
with num\_neighbors = 5 and using the simpliefied LOF algorithm, because
that is the default option with Sklearn. While many of the towns with
per capita crime rates greater than 40 still marked as outliers, several
were considered inliers because they had a similar enough LOF to 5 of
its neighbors. In contrast, many datapoints with low crime rates but
high home values considered outliers. This is because, compared to towns
of similar average home prices, those towns had higher crime rates.

    \begin{Verbatim}[commandchars=\\\{\}]
{\color{incolor}In [{\color{incolor}3}]:} \PY{n}{plt}\PY{o}{.}\PY{n}{figure}\PY{p}{(}\PY{n}{num}\PY{o}{=}\PY{l+m+mi}{3}\PY{p}{,} \PY{n}{figsize}\PY{o}{=}\PY{p}{(}\PY{l+m+mi}{10}\PY{p}{,} \PY{l+m+mi}{10}\PY{p}{)}\PY{p}{)}
        \PY{n}{plt}\PY{o}{.}\PY{n}{scatter}\PY{p}{(}\PY{n}{data}\PY{p}{[}\PY{n}{inliers\PYZus{}array}\PY{p}{]}\PY{o}{.}\PY{n}{loc}\PY{p}{[}\PY{p}{:}\PY{p}{,} \PY{l+s+s1}{\PYZsq{}}\PY{l+s+s1}{CRIM}\PY{l+s+s1}{\PYZsq{}}\PY{p}{]}\PY{p}{,} \PY{n}{data}\PY{p}{[}\PY{n}{inliers\PYZus{}array}\PY{p}{]}\PY{o}{.}\PY{n}{loc}\PY{p}{[}\PY{p}{:}\PY{p}{,} \PY{l+s+s1}{\PYZsq{}}\PY{l+s+s1}{MEDV}\PY{l+s+s1}{\PYZsq{}}\PY{p}{]}\PY{p}{,} \PY{n}{c}\PY{o}{=}\PY{l+s+s1}{\PYZsq{}}\PY{l+s+s1}{b}\PY{l+s+s1}{\PYZsq{}}\PY{p}{,} \PY{n}{marker}\PY{o}{=}\PY{l+s+s1}{\PYZsq{}}\PY{l+s+s1}{o}\PY{l+s+s1}{\PYZsq{}}\PY{p}{,} \PY{n}{s}\PY{o}{=}\PY{l+m+mi}{25}\PY{p}{,} \PY{n}{alpha}\PY{o}{=}\PY{o}{.}\PY{l+m+mi}{1}\PY{p}{,} \PY{n}{label}\PY{o}{=}\PY{l+s+s1}{\PYZsq{}}\PY{l+s+s1}{inliers}\PY{l+s+s1}{\PYZsq{}}\PY{p}{)}
        \PY{n}{plt}\PY{o}{.}\PY{n}{scatter}\PY{p}{(}\PY{n}{data}\PY{p}{[}\PY{n}{outliers\PYZus{}array}\PY{p}{]}\PY{o}{.}\PY{n}{loc}\PY{p}{[}\PY{p}{:}\PY{p}{,} \PY{l+s+s1}{\PYZsq{}}\PY{l+s+s1}{CRIM}\PY{l+s+s1}{\PYZsq{}}\PY{p}{]}\PY{p}{,} \PY{n}{data}\PY{p}{[}\PY{n}{outliers\PYZus{}array}\PY{p}{]}\PY{o}{.}\PY{n}{loc}\PY{p}{[}\PY{p}{:}\PY{p}{,} \PY{l+s+s1}{\PYZsq{}}\PY{l+s+s1}{MEDV}\PY{l+s+s1}{\PYZsq{}}\PY{p}{]}\PY{p}{,} \PY{n}{c}\PY{o}{=}\PY{l+s+s1}{\PYZsq{}}\PY{l+s+s1}{r}\PY{l+s+s1}{\PYZsq{}}\PY{p}{,} \PY{n}{marker}\PY{o}{=}\PY{l+s+s1}{\PYZsq{}}\PY{l+s+s1}{o}\PY{l+s+s1}{\PYZsq{}}\PY{p}{,} \PY{n}{s}\PY{o}{=}\PY{l+m+mi}{25}\PY{p}{,} \PY{n}{label}\PY{o}{=}\PY{l+s+s1}{\PYZsq{}}\PY{l+s+s1}{outliers}\PY{l+s+s1}{\PYZsq{}}\PY{p}{)}
        \PY{n}{plt}\PY{o}{.}\PY{n}{scatter}\PY{p}{(}\PY{n}{data}\PY{p}{[}\PY{n}{outliers\PYZus{}array}\PY{p}{]}\PY{o}{.}\PY{n}{loc}\PY{p}{[}\PY{p}{:}\PY{p}{,} \PY{l+s+s1}{\PYZsq{}}\PY{l+s+s1}{CRIM}\PY{l+s+s1}{\PYZsq{}}\PY{p}{]}\PY{p}{,} \PY{n}{data}\PY{p}{[}\PY{n}{outliers\PYZus{}array}\PY{p}{]}\PY{o}{.}\PY{n}{loc}\PY{p}{[}\PY{p}{:}\PY{p}{,} \PY{l+s+s1}{\PYZsq{}}\PY{l+s+s1}{MEDV}\PY{l+s+s1}{\PYZsq{}}\PY{p}{]}\PY{p}{,} \PY{n}{s}\PY{o}{=}\PY{l+m+mi}{1500} \PY{o}{*} \PY{n}{radius}\PY{p}{[}\PY{n}{outliers\PYZus{}array}\PY{p}{]}\PY{p}{,} \PY{n}{facecolors}\PY{o}{=}\PY{l+s+s1}{\PYZsq{}}\PY{l+s+s1}{none}\PY{l+s+s1}{\PYZsq{}}\PY{p}{,} \PY{n}{edgecolors}\PY{o}{=}\PY{l+s+s1}{\PYZsq{}}\PY{l+s+s1}{g}\PY{l+s+s1}{\PYZsq{}}\PY{p}{,} \PY{n}{label}\PY{o}{=}\PY{l+s+s1}{\PYZsq{}}\PY{l+s+s1}{proportion of LOF}\PY{l+s+s1}{\PYZsq{}}\PY{p}{)}
        
        
        \PY{n}{legend} \PY{o}{=} \PY{n}{plt}\PY{o}{.}\PY{n}{legend}\PY{p}{(}\PY{n}{loc}\PY{o}{=}\PY{l+s+s1}{\PYZsq{}}\PY{l+s+s1}{upper right}\PY{l+s+s1}{\PYZsq{}}\PY{p}{)}
        \PY{n}{legend}\PY{o}{.}\PY{n}{legendHandles}\PY{p}{[}\PY{l+m+mi}{0}\PY{p}{]}\PY{o}{.}\PY{n}{\PYZus{}sizes} \PY{o}{=} \PY{p}{[}\PY{l+m+mi}{100}\PY{p}{]}
        \PY{n}{legend}\PY{o}{.}\PY{n}{legendHandles}\PY{p}{[}\PY{l+m+mi}{1}\PY{p}{]}\PY{o}{.}\PY{n}{\PYZus{}sizes} \PY{o}{=} \PY{p}{[}\PY{l+m+mi}{100}\PY{p}{]}
        
        \PY{n}{plt}\PY{o}{.}\PY{n}{xlabel}\PY{p}{(}\PY{l+s+s2}{\PYZdq{}}\PY{l+s+s2}{per capita crime rate by town}\PY{l+s+s2}{\PYZdq{}}\PY{p}{)}
        \PY{n}{plt}\PY{o}{.}\PY{n}{ylabel}\PY{p}{(}\PY{l+s+s2}{\PYZdq{}}\PY{l+s+s2}{Median value of owner\PYZhy{}occupied homes in \PYZdl{}1000’s}\PY{l+s+s2}{\PYZdq{}}\PY{p}{)}
        \PY{n}{plt}\PY{o}{.}\PY{n}{show}\PY{p}{(}\PY{p}{)}
\end{Verbatim}


    \begin{center}
    \adjustimage{max size={0.9\linewidth}{0.9\paperheight}}{output_13_0.png}
    \end{center}
    { \hspace*{\fill} \\}
    
    This is the same graph as above but with green radii around the outliers
that are proportional to how much of an outlier that datapoint is. We
can see that even in the very dense area near the 0 crime rate
observations, there is still a datapoint that is considered more of an
outlier than even the datapoint all the way to the right of the graph.
This may be because simplified LOF is more sensitive to inliers or
because the towns of 0 crime rate have such similar home prices that
even a small deviation may make a datapoint an anomaly.

    The main advantage of LOF is that it recognizes local outliers that may
not be found by other anomaly detection algorithms. Because LOF is
always a ratio of the density of a point and the density of its
neighbors points, it will inherently be a local comparison. Other
outlier detection strategies may look for outliers using a global
average and will not find such local anomalies. Another advantage of LOF
is that it can be extended for use in high dimensional data. For
example, LOF was one of the highest performing algorithms when used to
detect network intrusion, for which there were 23 features to be
analyzed.5 LOF is also easy to apply to many different datasets.

The main disadvantage to LOF is that it is difficult to determine at
what value an LOF score is indicative of a datapoint being an outlier.
In some datasets, a score of 1.1 may mean it's a strong outlier, while
in others a score of 2 could mean that it is similar to other
observations throughout the dataset. This is the result of not comparing
data globally but is probably a common problem with unsupervised
learning techniques in general. Some ways to mitigate this are applying
LOF in an ensemble learning approach or other extensions of LOF, such as
Local Outlier Probability, which returns a value between 0 and 1 to
indicate the likelihood that an observation is an anomaly. Another
negative to LOF is that it can be computationally intensive. However,
this was mentioned more in earlier papers and a more recent reference to
the algorithm said that because of advances in hardware, this was no
longer an issue.

The Breunig paper concludes with some suggestions on how to expand LOF
and how to apply it correctly. They suggested looking at clusters whose
outliers were close to 1 or very far away from 1, to do more data
exploration on why certain clusters had higher or lower density. It was
also mentioned that in high dimension datasets, using different subsets
of the attributes may predict different datapoints to be outliers or
inliers. Both ideas suggest that LOF can be as useful in data
exploration as it is in anomaly detection.

    \paragraph{Footnotes}\label{footnotes}

1Markus M. Breunig, Hans-Peter Kriegel, Raymond T. Ng, Jörg Sander 2000,
3

2Markus M. Breunig, Hans-Peter Kriegel, Raymond T. Ng, Jörg Sander 2000,
3

3Markus M. Breunig, Hans-Peter Kriegel, Raymond T. Ng, Jörg Sander 2000,
4

4Markus M. Breunig, Hans-Peter Kriegel, Raymond T. Ng, Jörg Sander 2000,
4

5Lazarevic, A., Ozgur, A,; Ertoz, L., Srivastava, J., Kumar, V. 2003, 6

\paragraph{Bibliography}\label{bibliography}

Breunig, Markus M., et al. ``Lof.'' Proceedings of the 2000 ACM SIGMOD
International Conference on Management of Data - SIGMOD '00, 2000

Lazarevic, Aleksandar, et al. ``A Comparative Study of Anomaly Detection
Schemes in Network Intrusion Detection.'' Proceedings of the 2003 SIAM
International Conference on Data Mining, 2003, pp. 25--36.

Wenig, Phillip. ``Local Outlier Factor for Anomaly Detection -- Towards
Data Science.'' Towards Data Science, Towards Data Science, 5 Dec. 2018,
towardsdatascience.com/local-outlier-factor-for-anomaly-detection-cc0c770d2ebe.

``Local Outlier Factor.'' Wikipedia, Wikimedia Foundation, 25 Nov. 2018,
en.wikipedia.org/wiki/Local\_outlier\_factor.


    % Add a bibliography block to the postdoc
    
    
    
    \end{document}
